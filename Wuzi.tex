\documentclass{ctexart}
\title{{\LARGE《武子》选录}\\\textit{第一版\\勃民武学出版社$\cdot$武学思想部}}
\author{武冠轩}
\date{}
\begin{document}
	\maketitle
	\vspace{10em}
	\begin{center}
		{\Large 全世界武产阶级联合起来!}
	\end{center}

	\pagebreak

	\section{序}
		武学——新世纪的思想武器,为武产阶级的斗争而武装。
		二十一世纪是混乱的:没有了伟人思想的领导,没有了人类的本真,没有了人与人之间的真情实感与信任。武学是精神的桥梁,它为新一代青年提供了截然不同的,令人感到焕然一新的思考方式。抛弃传统哲学的意识形态,武学将人类本真,人类与异次元,人生与生存的意义用看不见,却又紧密的逻辑思路连接,在批判陈旧的哲学思想的同时,利用其糟粕,形成了不可被反驳的思考方式。

		随着互联网的发展与普及,新文化,新观念与新思想触手可及。武学在审视历史的过程中,跟进着时代的步伐,从人际关系,市场经济,政治哲学等方面对社会的形态进行定位与评价。与随波逐流的大脑降级人不同,武学思维方式缜密,不随波逐流。它用严谨的思路将常人的观念颠覆,提出全新的看法,如“婚姻是合法的长期卖淫”,二次元与资本主义之联系等先进思想,是二十一世纪青年人应当关注,并且深入学习的。新的意识形态能够帮助新一代人类快速进步,从而取得更多的人文,科技以及社会的进步。

		本书将侧重与武学创世人——武冠轩的代表性言论以及代表性新思想文章。这些文字材料记录了武学在盛世时期的发展以及在暴政与霸权压迫下,奋力反抗的武学思想。

		致敬武学,大脑升级,戒掉呼吸,向死而生!
	
	\pagebreak
	\tableofcontents
	\pagebreak
	\section{武学的精神家园:文明创国元姥院}
	\subsection{互联网“键盘政治圈”的实质——随想}
	近期,在对“天朝渣男图鉴”内容的正常讨论的过程中,一位朋友通过po出来鄙人在贴吧发的“黑历史”“男权主义”帖子来指鄙人为完全的“男权主义者”,以此试图证伪鄙人当时正常讨论的内容;且不说这种挪移和戴帽子的行为是否在正常讨论当中作为驳倒他人的方法是否合理,就只说我这个“钓鱼贴”本身,以及千千万万的更多的“钓鱼贴”的产生,来分析一下其前因后果。
	
	\begin{itemize}
	
	\item[一、]我的罪过,但是是你的罪过在先:乌合之众——钓鱼行为的原因和产生
	
	\begin{itemize}
	\item[$1.$]  群体理性的缺失让温和的人走向极端
	
	如社会心理学理论所言,“影响大众想象力的,并不是事实本身,而是传播和扩展的方式”,当一个观点开始扩张之后,很快就会出现其中的激进派和温和派,但是常常激进派的观点在其中会胜出。“人生不如意十八九”,大部分人的生存境遇并不是精英的境遇,而这个群体总是善于找到各种各样的钳制自己发展的外因,比如对于本民族的人来说这个外因是“外民族侵占了我们的生存空间”,这时候一夫振臂则万人山呼万岁,于是有了纳粹;比如中国以目前的发展进程一般人的生活质量与美日欧发达国家相差还很远,这个外因就被找到了“西方对中国经济阴谋”或者是“历史上美日欧对中国的侵略”,于是有了高举极端国家主义旗帜的疯狂的“愤青”和“兔杂”;比如女性在封建社会和早期资本主义社会一直以来都被物化,被作为男人的附庸物存在,缺乏人格自由权和公民权,且由于世界人口基数的庞大这种腐旧的现象在全世界仍然广泛存在,于是一部分女性就开始要求清算“直男癌”、“渣男”,于是有了极端女权主义者。这少数的极端分子所言很快引发了其他受压迫或者曾经受压迫者的共鸣,很快其中的异见也就被“求同存异”掉了,为了求同,这部分异见者也逐渐放弃了自己的异见——因为一切异见都是斗争的阻碍因素。而还有剩余的一部分异见者,以及一些原本的局外同情者,却被打到了敌人一方。以这些少数温和派异见者的势力,自然无法和极端派以及原来的斗争对象同时抗衡;所以就不由自主开始倾向于原来的斗争对象方向,或者说是对自己原来的“战友”的怨恨已经超过了对敌人的怨恨,于是就有了这些所谓了“叛变者”。而对于那大多数,由于响应已经成了习惯,而其中大多数也没有理性思考的习惯或者能力——就选择了各种各样更极端、更非理性的理论。
	
	\item[$2.$] 以其人之道还治其人之身——天然的复仇主义
	
	本质而言,上述的极端思想是出于复仇主义,也就是:
	
	他们对我们做了什么,我们也要对他们做什么,甚至我们要做的更狠,才能体现我们的光荣。
	
	这确实是一件痛快的事情,古今中外的各种文学艺术作品没有少讴歌过这样的英雄;但是当所有人都去效仿呢?
	
	比如说,柯蒂斯李梅,二战美国空军将领,以二战末期对日本的本土空袭“李梅火攻”而著名;不得不说李梅火攻大大加速了日本军国主义政权的垮台;日本军国主义的暴行有目共睹,其对中国、苏联以及其他同盟国的民众的残忍行为和李梅相比是有过之而无不及,自然在战争结束时,李梅是一个英雄。但是,在战争结束后,战时这样的行为我们自然可以评价为战争所需、必要之恶;可是在和平年代如果谁要提出复制李梅火攻,那么不是反人类,是什么呢?可惜,还真有的人这样做:
	
	“宁可中国不长草,也要杀光日本人!”——$2013$年时中国部分‘爱国民众’的游行标语
	
	“日本是有罪的,其中没有一个人无罪,他们并没有付出应有的代价,我们必须要消灭日本人,抵制一切日本的…….(以下省略)” ——易梓嘉(B站ID:日本-有罪,东京李梅烧烤)
	
	“我们没资格替死去的老一辈原谅日本人,现在我们就要送他们去见死去的老一辈” ——QQ看点评论区
	
	自然很快就有人对这种理论看不上眼了;当然这些人往往也是不受主流待见的,比如2013年对于砸日车行为的网上调查当中竟然有55\%的网民对此持支持态度;所以这些人就也根据一般的传播和扩展方式,选择了相反的方向,他们开始称呼这些同胞为“支那豚”,开始使用“你国”一类词汇,开始仇视自己的民族,走逆向民族主义的道路,其实这些人究竟如何呢?
	
	鄙人知道一个Twitter上的著名“真支黑”,ID叫做 @橘豚,这个人常常创作“辱华漫画”,也就是画一系列专门反映国人丑态的漫画,其中国人的面目被画成了猪头,而其他国家的都是人;这种行为自然是典型的“精日”行为。但是,在鄙人仔细看了“橘豚”的作品和一些个人资料之后,我能看出来,这个人最初大概也只是想揭露中国少数人的行为的,目的是为了让更多的国人引以为戒;但是由于不堪“小粉红”的围攻而逐渐转向了极端。
	
	女权运动也是一样的,在温和派逐渐不堪激进派的步步紧逼之后,常常破罐破摔,干脆和“男权”站在同一阵线,去对抗自己原来的“战友”。每个人的耐心都是有极限的;试想一下,一个温和派只是因为其提出的相对妥协或者均衡的理论而被激进派的战友围攻的时候,尤其是当受到对方的非常规辩论——偷换概念、强制逻辑、甚至疯狂的辱骂,这时候又能怎么样呢?——骂回去吧。——形成恶性循环。
	
	\item[$3.$] 新生事物内部矛盾更加尖锐,而越腐旧的事物越是铁板一块
	
	对于新生事物来说,其中不可避免的分出了各种各样的派系:虽然他们的思想都是希望走向更好的一面,但是其各种方面上,比如方式方法上、组织结构上等都是有不同的,且这种不同有时候是致命的。但是,因为我们暂时还没有足够的经验证明哪种是正确的,这就意味着有一个摸索的过程,其中免不了辩论,所谓“真理不怕辩论,怕辩论就不是真理”,在不断辩论、实践之中,逐渐摸索出来最好的方案。但是激进派的“自动主流化”让辩论很多时候被压抑了下来。于是就导致了不断的尝试与错误,历史也就这样波浪式前进、螺旋式上升。比如雅各宾派,就是典型的激进派,他们是当年要求谋取自由最为激进的派别;但是最后其统治被评价为“恐怖政权”,其结局也都很不好,处决路易十六的断头台最终处死了罗伯斯庇尔。所以,激进的试图加速,事实上延缓了历史进步的进程;因为对方也会用一样的复仇主义来反扑,甚至这样的复仇主义者还可能是自己原来的战友,罗伯斯庇尔被处死的时候也响起了$15$分钟的掌声。
	\end{itemize}
	\item[二、]互联网本质:人类的本质就是一个戴着面具的复读机
	\begin{enumerate}
	\item 在互联网上,你可以随时修改自己的“观点”“意识形态”,一切以自己的心情为前提
	
	很多人混在网上的一些“圈子”里面而并没有对这种“圈子”的实质有任何的理解。很多人以为,网上一个人的表现就是现实一个人的表现;但是很显然这完全是一个伪命题。
	
	鄙人曾经在网上混过“左圈”,也就是“左派圈子”,这个“左派”也着实是一个笼统的概念,包括了各种各样的毛派、托派、铁托派甚至霍查派,但是他们都有一个共同特点——网络和现实基本关联不大。在现实中,这些人当中的大多数和其他没有什么特别政治观点的人也是极其相似;他们可能有时候会“线下融工”,但是他们的共同特点就是不会在线下去做什么上纲上线的事情。这些人是真正看清互联网的人;但是也有一部分人,把网上道听途说懂得的一些所谓道理奉为圭臬之后,将其作为自己的“唯一事业”,比如某“莱茵学社”真的在将网上网友的一些言论整理成文章、写成杂志,然后一本15元的价格向外兜售,这种行为并没有为他们带来丝毫的利益,除了被人耻笑和唾弃。
	
	记得在$2015$年,鄙人刚刚开始混各种键盘政治圈(左圈),也是曾经立志要在现实实现真正的共产主义,但当时却群里面有一个“阴阳怪气”的人,忽左忽右不断转换自己的意识形态。后来去追问此人,此人却说:“你终于有一天会明白,这一切都是为了找乐子”,当时我不相信,但是后来在我看了一下“网斗尼亚”们的行为之后,鄙人也算是看穿了——所谓的网络斗争,不过就是一群人在语c,在臆想自己如果是决策者能怎么样——如果说到现实中的斗争,没有一个人会去支持,而凡是把网络言论放进现实的默默无闻的普通人,几乎都是在网上被视为乐子的人。到后来,鄙人的行为果然难逃此谶:哪些键政人惹我生气,我和哪些键政人对着干;哪些键政人过于搞笑,我去故意刺激这些键政人给自己找乐子。很快我发现似乎一半的键政圈的人都是这个样子。他们自然也有他们的基本原则,也就是基本的为善的价值观,在现实中。但是在网络上,我们这些人却表现为所谓的“网络恶流”,为什么呢?上面说到的复仇主义思想与娱乐至死的思想联合作用的结果。
	
	\item 键政和病毒营销:抓住人图乐子和跟风的特性
	
	“转发这只锦鲤….”“测测你的…..”这些无聊的东西,目前在各个社交媒体都层出不穷;国内的微信朋友圈、QQ空间、微博,国外的fb、Twitter;要问有没有人信这些东西,鄙人认为,只能说“无法排除存在”,但是对于大多数来说,他们是不信的。不信为什么要转呢?——没有损失嘛!转了万一管用呢?——于是我们莫名其妙就被各种各样绑定广告的量产的各种“测试”包围,被各种各样的“转发这个….”的炒作蹭热度的内容包围,但是很多时候我们还是忍不住按下右下角的转发键。
	
	同样的,由于图乐子和跟风两大特性,越来越多的人开始在键政圈里面表现极端,他们极端的言论很多时候令人作呕;但是这些人却越陷越深,直到有一天突然发现其实他旁边的人也是戴着面具的人。尤其是在实质上,利用互联网的言论而得到现实注意的概率微乎其微,所以越来越多的老人转为用网络玩玩的同时,越来越多的萌新入坑并转向极端。反正,在网上无论说多恶心的话语,都是网上,都是藏在一个面具背后;隔着网线砍人,除了像722文登事件那样子“极端派对阵极端派”的行为以外,可以约等于零。所以,逐渐的,网上的言论也越发虚假,对于键政这种并不需要什么实质根据只在于发言的地方来说更是随意胡说。用种族主义思想足以气一气当真的键政左,用极权社会主义思想也可以气一气当真的键政右,气你,我高兴,这就是键政的实际逻辑。没有损失、跟风、然后取乐子,这就是网络言论的实质。或者从各位更加熟悉的方面来讲,百度贴吧那么多“反XX”吧,有的反某个游戏,有的反某个文学作品,有的反某个人,但是他们真的是要将其置之死地的反吗?有,但是很少;绝大多数的人,反的是脑残粉,而且事实上也并不是直接说“反的是脑残粉”,因为这毫无意义,他们选择用毫无逻辑的方式钓鱼,从而将脑残粉的围攻作为乐子而已。
	
	\item 人心的形而上学:多重思想和变化思想
	
	乔治奥威尔的小说《1984》里,有一个概念叫做“双重思想”,也就是一个人的脑子同时有接受两种思想的能力;这看似只是在英社党的极权统治下的权宜之计,但是事实上在这个世界上是广泛存在的。网络的匿名性,以及网上言论对任何人的物质极少造成损失的特性,让网络成为了如此的一个多重思想的聚集地。在互联网上,放之四海而皆准的规则就是“能用什么思想找乐子,就用什么思想找乐子”;在这里,人甚至可能故意去读自己所厌恶的一些相关内容,故意说出来自己也并不喜欢的言论,但是这种自己也并不喜欢甚至厌恶的内容并不会对自己构成任何严重的损失——隔着互联网的巨大幕布,你就可以享受一下钓鱼的快感,甚至还能钓起来几个“恰巧真的信你说的话”的人,这样的情形经常让人忍不住笑。林子大了什么鸟都有,而网络就是这么一片林子,这样子无非就是捕鸟。
	
	但是,在大多数人认为,人仍然是只有一面的,或者说只有一面是真的,其他面都是假的,而这一面就是自己最痛恨的一面。并且这一面必然也是不会变的:一个人十年前的思想和现在的思想必然是一致的,这种现象是普遍的,极少有人能够避免这样的成见。
	\end{enumerate}
	\item[三、]必须承认:即使对于不愿意换位思考的人来说,不去换位思考,带来恶性循环,也是不好的
	
	就事论事,从后果上来讲,鄙人的钓鱼行为是绝对错误的;这也是鄙人在(一)当中说过的复仇盲动的体现。鄙人钓鱼之前,被一大群极端女权主义者的思想所激怒;这些人的思想包括:
	\begin{enumerate}
		\item 	男性要无条件服从和忠诚于女性,要对女性的各种不合理的所谓“性子”来包容和忍让,因为这是女性的本质——一方面积极的女权主义反对封建的极端本质主义思想,另一方面这些人却又搬出来了所谓本质来试图化女权为女尊,孰之过也?
		\item “直男会知道给女生买圣罗兰口红,色号要XXX,凡是不愿意多花钱买奢侈品送女朋友的都是直男癌”--请问说出这样话的女同学,你是喜欢的男生这个人,还只是喜欢他的钱包和你的虚荣?不愿意关注你的人自然不好,但是关注你就意味着必须要把自己的经济权利取消吗?
		\item “男生负责照顾好女生,要给女生足够的物质依靠的同时为女生做所有事,包括做家务等”——请问你是要男友还是男仆?
		\item $\cdots$
	\end{enumerate}
	
	在非理性人的面前,逐渐走向非理性,这是鄙人的过错;鄙人的错在于没有仔细想这段话的攻击对象而打击到了女权派中的温和派,如果看这篇文章的有真正的温和派,真正支持平等、要求在于平等独立人格和权利的朋友,鄙人愿意非常诚挚地为您道歉。
	
	毕竟,鄙人这种图一时之痛快的行为,事实上加剧了温和走向极端的过程,是鄙人之过也。
	
	结语:“置之一笑阿尔泰、信以为真则马来”--键盘斗争中的态度
	
	鄙人有段时间曾经接触过一种键政圈当中常用的所谓“理论”--泰马学。泰马学算是一种变体的种族主义,最早搞出来这些概念的是“匪贼鸥”之类的网友,简单来说就是汉人很多都被脑容量低下的劣等马来取代。但是这个理论的意义是什么呢?在我问过几个玩“泰马学”的网友之后,得到的结论无非就是三个字:“辱蝗汉”。(注:蝗汉,也就是大汉族主义者,属于一种极端民族主义思想,其思想在于将少数民族赶尽杀绝、建立种族纯粹的大汉族政权。)后来鄙人真的去看了老匪这个“泰马学鼻祖”的社交媒体,才看到了这句话:
	
	置之一笑阿尔泰、信以为真则马来
	
	这大概就是键盘斗争当中遇到各种各样奇怪观点的最好态度了吧。
	\end{itemize}

	\subsection{你姨有一点说的倒是没错}
	
	民族都是发明的,所以请你匪立即停止发明民族,鼓吹你妈的民族主义呢?一群弱智一样。
	
	\subsection{远离知乎}
	
	支忽垃圾网站已经无可救药,7天,又是7天,你封禁你大爹的时候可真是客观公正(呲牙)。
	
	本人之后的一切指点江山一切黑屁全部在博客更新,基本不在支忽活动,qqqxx
	
	
	\section{时代的最强音:武学女权}
	
	\subsection{辩女权}
	
	最近一段时间,在 @Yihong 泓老师的带领下,我开始在X乎键政圈关注各种各样的女权/反女权团体,逐渐的感觉整个中国互联网的键政圈比我想象的更加恶臭,两派同样恶臭的派别在互相攻击当中滑向三个同样反动的极端——极端女权主义(费米纳粹)、实质上维护父权制的伪女权主义和传统主义。
	
	费米纳粹们在X乎的政治正确,自然是被批判、被耻笑、被辱骂的对象。对于这样一个极端而又事实上反动的,盘踞在微博和贴吧的键政圈的群体,我个人也是非常反对的;但是我们需要分析这样的人的出现原因。
	
	这样的人无非两种:蠢或者坏。我不是个儒雅随和会圆滑说话的人,我激进、直白,我就是这样子。第一种人,所谓的“蠢”,这里指的是心理有问题的人。这些人中一部分,因为社会的构建,比如自己身受或者目睹过多父权制导致的痛苦和悲哀,变的有一些类似于PTSD的倾向,也就是仇男。这种人我承认存在,我愿意理解,但是我不会去支持。还有一部分,则是受到其他少数人的影响,加之自己作为极少数的社会偏离值存在,而开始产生这样的思想。我不会对这些人冷嘲热讽攻击炒作,虽然一些批判,或者说是所谓“警醒”的希望是存在的。还有另一种人,那就是纯粹的坏,真正的别有用心;这些人的目的也是两种:一是通过炒作来赚取热度然后热度变现盈利发互联网传媒的脏财,二是通过极端言论引起注意钓鱼来污名化女权让更多的人滑向另一边。这种人才是最可恨的,是平等权利和社会进步的重大威胁。
	
	上述这些人类似于以利亚·穆罕默德在黑人民权运动当中的地位,甚至还不如。他们(注意,我使用了“他们”一词,因为这些人在键盘背后我们并不知道其真实性别,尤其是第二种人的存在)至少在表面上追寻的,犹如“黑人至上主义”在种族隔离时代的美国之于“黑人民权运动”,通过敌对、仇恨<b>最普遍的</b>男性来达到其所谓“女权”的政治目的,但是因为这种仇恨的群众基础过小(占人口数量一半的女性当中的少数),终究止增笑耳,除了污名化女权主义以外,连女性觉醒都没有办法做出比较明显的贡献。
	
	然后再说第二种人,这也是X乎键政圈普遍批判的一种,这种人的典型就是某些街头采访里面表现出来无上的拜金的那些女人,比如“男朋友必须有100万才能{\bf 养}我”。这种反动腐朽落后的封建-资本主义杂交的奇特思想被部分X乎键政人视为最典型的“女权”。殊不知,事实上就里面这个“养”字,这些人的本质已经暴露——无非就是想要继续将自己物化,让自己成为类似于宠物的存在,继续存在于传统主义的家庭当中,不要社会权利和社会责任。这种行为,完全和“女权”搭不上边,只能说是被拜金主义普遍存在的社会蒙蔽太深了,以至于自己成为商品而不自知。男性同样也有一样的问题,最典型的就是各种各样的攀比炫耀的行为,Supreme、AJ这些智商税在全社会的火爆证明了一切。在资本主义社会,每个人都被异化,如果说这种被异化的表现也是女权主义,那么这样的女权主义还真的罪该万死,早死早超生。这样的“女权”在之于真正的完全平等,是完全反动的,因为这也就意味着所谓的优势基因才能传承,事实上还是一种社达的体现。
	
	再说说传统主义者。这种人又叫做保守主义者、右翼,是一种自然存在的群体,也是社会变革中的“电感”。这些人非常符合“楞次定律”,无非就是社会有任何可能有风险、可能夹带危险因素的重大变革,无论其是进步还是退步,都会阻碍,只希望能够守好自己的一亩三分地,过好自己的小日子。这些人,除了少数完全的既得利益者以外,统一的特点就是--{\bf 信奉“常识”}。这里说的“常识”,指的就是目前的社会现状给这些人的思想纲领:现在的家庭是“男主外女主内”,那么“男主外女主内”就是最合理的;现在的生产方式是这样,所以代替人工的机器人要质疑;现在的社会是资本家通过对资本的控制来经营工商业,所以这样的资本主义社会要持续万万年。这些人还有一种叫法,叫做“{\bf 小农思维}”。中国毕竟总体还是一个农业文化为主的
	国家,无论是在座哪位X乎精英,族谱里后序遍历一波,总能在5代之内找到一个以上农民出身。农民的特点就是保守,因为农民拥有自己自耕的土地和自给自足的经济基础,进行任何地理上的迁移的难度都很高,所以接触的圈子也是小国寡民的圈子,所以有任何异端思想的出现都会被抬头不见低头见的各种老乡排斥。一个典型例子就是美国南方的红脖子们,这些人虽然在黑奴解放之后既得利益被剥夺,但他们仍然是农民或者半农民(居住在郊区、祖传房屋、搬迁花费巨大、家族产业),每次投票仍然要一股脑投给中间偏右翼的共和党,希望看到一个没有这么多变革的社会。事实上,根据人们对这些人的了解,其实这些人内心善良淳朴,很像我国最一般的、上了年纪的农民。(给杠精们看:不要用快手那种东西说事,快手那种属于城乡结合部的农民工等,属于疯狂自我炒作的商业社会异化产物;就大多数传统的农民,确实是这样的,保守、但是淳朴善良)
	
	对于这些人来说,孔儒、宗教这样子的信奉了多年的、一直以来也以“善”为主旋律的思想是主流,但是这些东西当中的糟粕也保留的非常之多。即使是在2018年,用1918、1818甚至1018年的思想想问题的人不在少数;事实上,不用说1018或者1818,即使是1918,达到完美的男女平等也是不可能的,因为普遍的工业还需要劳动密集,还需要大量的体力,而男性天然优于女性的也就只有体力了。(这也是少数现代男性沙文主义者提出的“两性近似偏男性”需要“温和父权”“维持K策略人口增长”的来源,或者说是“同样搬砖男性就是比女性搬得多”这种论调。)但是在2018年,以至于未来我们将会看到的2038年、2058年、2118年,情况会怎么样呢?越来越发达的生产力正在越来越多的解放每个人的体力。
	
	我们知道,经济基础决定上层建筑,在只有牛耕的时代,能产生的也只有地主、农民之间的封建租佃关系。欧洲资本主义起源于耕作方式的增强以及新大陆发现之后的新作物,从土地耕作之中解放出了更多劳动力,所以很多人说文艺复兴、启蒙运动是“老外闲的没事干”,也确实没什么问题;在农村被养着闲的没事干的情况下,他们会去发展城市工商业,毕竟农业已经没办法为他们赚更多钱,于是进一步提高生产力,当旧的封建制度开始制约发展的时候,从反对的声音开始,到反对的实践,这些新的利益诉求者就会推动封建社会变成具有大量封建残余的资本主义社会,一直到两次工业革命将封建残余的部分逐渐击碎。接着,随着资本主义社会不公平在越来越发达的生产力下越来越突出,劳动力逐渐被机器自动化替代,尤其是繁重的、基本不需要脑力的体力劳动。体力劳动的减少使得社会不再非常依赖体力,这样子更多人就希望通过脑力进行发展;然而脑力发展的机会却不一样,这就导致了机会平等的需求,所以,目前社会最优解其实是“{\bf 具有大量资本主义残余的社会主义}”,也就是说,无产阶级保障机会平等,而在一定程度上保留社会竞争。这种情况下的社会竞争则主要体现在脑力上,社会贡献也是通过脑力进行体现。而男女的脑力没有本质差别,这种情况下,父权制就跟着传统资本主义、封建残余一起失去了市场。但是在农民之中呢?他们的生产生活一直没有改变,那些从农民当中脱出成为工商业者的人也已经被“开除农民籍”,可能他们多了几个拖拉机开,但是总体来说他们还是只能固守土地房屋等不动产,所以他们的思想在小圈子里面保持不变、落后于时代是再正常不过的。
	
	事实上,我一直提倡摒弃“女权主义”这个概念,一是因为现今伪女权们对女权主义的污名化,二是因为“女权主义”这个词汇具有很强的局限性。我更提倡用“社会平等主义”或者类似的词汇来代替这个词。原因如下:
	
	\begin{enumerate}
		
		\item “女权主义”历史使命已经完成,现代的平权要求应该和早期的“女权主义”区分开来。传统上的“女权主义”,局限于资产阶级内部,对无产阶级的裨益有但是并不是很大,因为无产阶级女性早已经走上了工作岗位,只不过一直没有做到同工同酬;女性一直没有获得投票权,但是当年的投票权事实上是让资本为主、无产阶级只能在两个以上资产阶级“主子”当中选择其一。当时的女权主义,事实上是资产阶级革命的一部分,是对封建传统主义的一次革命;而现在的平权,是无产阶级革命的一部分,因为,性别的平等从来都不是独立在其他的平等之外的,想想看,有的种族也在受到歧视,有的人还在因为父母的经历而被问罪,更重要的,有很多人还在传统封建-资本主义杂交的家庭家长制当中被迫去婚恋,更更重要的是,{\bf 有很多人还拿不到他们劳动的价值,而被迫自我物化,把自己变成一个机器。}现代的社会的彻底的平权,不仅仅是对女性的解放,也是对男性的解放,是对全体受剥削、被资本主义异化的人的解放。
		
		\item “女权主义”的流派过于复杂,导致其中鱼龙混杂过于严重。上面我提到的两种——费米纳粹和伪女权——都是本质上反动的。如果继续打着“女权主义”的旗号来吸引这些人加入,那就很像第二国际当年让一群支持帝国主义侵略战争的“社会主义者”加入了。如果不和这些反动的分支划清界限,那么只会把更多普通男性推到另一个方向,推到敌人的方面,而不能达到我们要的平等权利的效果。
		
		\item 片面强调“女权”会导致人们认为现在的“女权”还是19-20世纪“女权”的延续,这导致更多人成为我们的敌人。如上文所讲,19-20世纪的女权主要集中在平等公民权、经济权这几个方面,而在全世界的大多数地方,公民和经济意义上的男女平等已经在立法上做到了,所以很多人就会认为我们是要搞“女尊”,要让女性投票“一票顶三票”,要让女性“多休假拿多钱”,这是完全的一派胡言,泼脏水。我们是要在社会各个方面彻底取消任何人看任何性别时的有色眼镜,让所有人对所有人一视同仁;绝对不能让这样的歧义误解延续下去。
		
	\end{enumerate}
	
	\section{武学的崛起——武一世$\cdot$失败人13}
	
	\subsection{作为亚文化的二次元在资本主义社会中起什么作用?}
	
	谢邀。首先本人就是“某网站(反萌网http://www.fanmeng.org)”站长;首先总体说就是关于文艺的阶级性。具体文艺绝不是脱离阶级存在的,而萌文化就是现在非常强势、不断向下流动的一种资产阶级文艺。这里首先要牵扯到一个概念就是ACGN。ACGN其实是一个很模糊的概念,并没有一个完整唯一的定义,但所有说法当中大概两个是主流:一是所有动漫(Animations),漫画(Comics),游戏(Games),小说(Novels)的总称,这也就是一种文化形式,是无可厚非的,所以一部分人把这边的反萌视为反对整个ACGN的文化形式是一种误读。还有另一种定义,这就是要被反对的了。另一种定义:ACGN,即萌系动漫、萌系漫画(本子)、galgames、同人及轻小说。这个就绝非一种文化形式了,而是一种具体的文艺,其阶级是资产阶级。事实上,资产阶级搞萌文化自然是有其目的的,毕竟这些人以维护自身的利益为目的。可以说,如果没有无产阶级施压,资本家就无所不为;因此自从雇佣关系出现,就有着资无矛盾,矛盾激化到一定程度,无产阶级必然会造反,小则罢工游行,大则暴力革命,这就一下子动摇甚至终结了资产阶级的剥削。因此资本家也是不敢去激化资无矛盾的,相反他们会在这样的压力下被迫缓和,资本家毕竟也懂“细水长流”,少剥削点,让更多工人成了精神小资,从此减少要求,岂不美哉。于是资本家便开始用各种方法来欺骗工人达到一种暂时的缓和,大概有这么几种手段:第一是小恩小惠,比如福利等。没错,这是工人阶级斗争的结果,其实更确切说应该是资本家迫于工人的压力被迫做出的让步。这些不会动摇资产阶级的统治,但是总是让资产阶级受了些损失;于是资本家想出了第二种手段,转移。转移,很好理解,说白了讲就是给工人吊个沙袋,让工人去打沙袋不就不打资本家了。这个沙袋有很多形式,比如外民族、其他国家、竞争对手、少数道德败坏者等。将主要矛盾转移开,这样子确实可以暂时分散无产阶级的注意力,并且确实奏效,但是因为现在资本的全球化,利用民族主义来转移矛盾也快要变成一种“伤人一千自损八百”的行为了,而利用少数人转移矛盾,少数人太少了。于是资本家想到了第三种办法——麻痹。而这个麻痹也是形式众多,包括现在的成功学、鸡汤,也包括了萌文化。(现在先写到这里,先去上课,一会补充。)
	
	\subsection{中革中央到底是什么?}
	
	类似于民意党的组织,本身坚持无产阶级专政和无产阶级革命是对的,但是他们犯了“血债血偿”这种明显的血统论错误,还有就是否认当前赵国的帝国主义实质,搞布朗基式的暗杀行动,不是一个真正意义上的无产阶级组织。
	
	\subsection{理性分析:为何宅中多左派,少女尤爱国?}
	
	谁告诉你的十宅九左我看到的宅豚多数是田园右废物和精赵小粉红。小粉红绝不是什么“左”,而是极右,是法西斯的拥护者。所谓的“宅左”也是个虚有概念,你说一群宅豚NEET啃姥小资产阶级废物还能算什么左派,不融工、不读书、不团结的“三不原则”倒是所谓“宅左”常见的。一群伪左,还不是情怀党。我倒不是自我标榜我自己的“反萌”立场多么左派或者是我自己是如何左,我也是一小资废物,但是起码我不会自我标榜是什么狗屁“左”,你国的田园左还不都是些废物。至于少女爱国,我只能说,一群小资产阶级当然不知道你国体制在榨干多少底层无产阶级的血汗。
	
	\subsection{知乎左圈有哪些用户?为什么天天在开除左籍?}
	
	佐juan,一群小资产阶级语c意淫之地,天天瞎几把键盘斗争屁用没有。
	
	\subsection{如何看待 B 站上有人鬼畜国际歌遭指责的现象?}
	
	屑站小资产阶级日常,反正他们也不当回事。
	
	\subsection{共产主义支持性解放吗?}
	
	泻药。我不知道我一个佐juan小资废物怎么就被邀请到这里来黑屁了。
	
	我不知道《美丽新世界》这样的反乌托邦、超帝国主义(考茨基大佐的理论)怎么就算共产主义了。但是我可以肯定地告诉你,{\textbf 共产主义不等于共产共妻。}
	
	女性的解放是和社会主义革命一体的,共产主义要追求的也是每个人实际上的完全平等。但是这不是说要“共产共妻”。
	
	看楼下的某人似乎又开始放女性道德的黑屁,我也不能说什么。但是,性方面的压抑确实是阶级社会的产物。共产主义是支持性解放的;但是这个性解放不是你们所想的那样“滥交”,而是有关乎性的话题不再敏感,可以正常化。
	
	其实我理论水平也很低,欢迎批驳。
	
	以上。
	
	\subsection{资本全球化是否可以助推无产阶级革命与共产主义发展?}
	
	泻药。
	
	我个人知识水平低的可怜,所以只能发表一些黑屁暴论。
	
	资本全球化,首先我们对于不同国家的人分类讨论,对于那些发达国家、帝国主义大国,无产阶级也分享到了帝国主义剥削的红利(虽然相比于跨国垄断资本家来说是九牛一毛),斗争性反而会相应降低;更多的斗争体现为非暴力不合作式的工会改良运动。
	
	对于那些欠发达国家,无产阶级被剥削更加惨重。他们是全球化的受害者,成为半殖民地。他们可能会组织起来一些民族主义运动,但是很快这种由民族资本家领导的运动又因为帝国主义的制裁、社会主义强国的缺失,要倒向帝国主义国家,也达不到革命的效果。非洲中东很多国家就是在这个情况下。
	
	无产阶级革命仍然是最有可能先发生在帝国主义的最薄弱一环,类似于20世纪初期的俄国那种。
	
	以上。
	
	\subsection{一个医生如何快速合法完成原始资本累积?}
	
	泻药。
	
	如你所知,原始资本积累就是将生产者和生产资料强制性地剥离,也就是说压榨掉别人可以用于购买生产资料的钱财就行了。
	
	你应该懂得,开高价药,吃回扣,花式逼着病人掏大价钱,你搞到钱,那就原始资本积累了。
	
	合不合法?你国法律反正是人定的。
	
	以上。
	
	\subsection{性取向属于上层建筑吗?}
	
	泻药。
	
	性取向个人认为属于一个个人特性的;虽然我也无法确定性取向是先天还是后天,但是根据一般基因上的解释来说,先天同性恋不大可能将他们的基因保留下来的。
	
	其实因为一个人性取向的形成很难说明,也很难分析究竟是先天个人原因还是社会原因。
	
	但是,无论如何,一个人的性取向怎么样也不是全社会的性取向;相反,这种生物学的东西大概和社会关系不大。上层建筑更应该是一种社会制度性的东西,比如说对LGBT的政策。
	
	我水平很低,欢迎批驳。
	
	以上。
	
	\subsection{既然共产主义是历史必然,我们为什么还要为止努力?共产主义和社会主义有什么区别?}
	
	泻药。
	
	共产主义并不是所谓的“历史必然”,应该说如果继续发展下去必然能达到共产主义,但是如果停止发展陷入类似美丽新世界的反乌托邦或者是世界核战争毁灭呢?——那就不一定了。所以仍然要为共产主义而奋斗。继续发展下去,就需要人的主观能动性,其中之一就是社会制度的发展,也就是暴力革命。因为没有革命推翻资产阶级,资本主义生产关系迟早有一天会成为生产力发展的阻碍因素,因为资本主义本身就是难以为继的,竭泽而渔的。你思考一下,价值不断从无产阶级手中被剥夺,流到少数资本家的口袋里,无产阶级的消费力不断降低,迟早会经济危机。但是不一定这种经济危机就意味着共产主义自然而然了,工人自然而然造反了。相反,资产阶级会寻求自救,而很多时候自救方法则趋向极端,比如纳粹不就是为了拯救德国的资本主义吗?
	
	共产主义和社会主义的区别在于社会主义还是有阶级的社会,是无产阶级专政,无产阶级掌控国家机器,实行按劳分配,这也就意味着很多资产阶级法权还没有办法消失,原因是生产力的落后。生产力不足以实现按需分配的时候,社会的最优解就是按劳分配;按劳分配——社会主义,同时社会条件还允许资产阶级的产生,因此仍然需要国家机器来对资产阶级进行专政。但是共产主义下,社会生产力高度发达,各尽所能、各取所需,每个人都拥有充分的自由和机会,资本主义复辟的可能性也就会消失,因为没人希望恢复剥削制度了。在这种情况下,也就不需要政府(或者说不需要国家机器,可能会保留类似于一些管理和进行一些规划的机构),不需要军队监狱警察等了,因为全社会每个人都没有任何造反的必要了。当然社会不会没有矛盾;所以说共产主义并不意味着社会进步的终结,但是共产主义的矛盾我们因为远远达不到,还未曾知晓。
	
	以上。
	
	\subsection{你为什么支持死刑?}
	
	理论上来说,凡罪皆死刑是最好的,因为既然犯罪就意味着这个人本质上对社会是负贡献,根据社会达尔文主义的理论这种人就应该被彻底取消存在权,所以说应该凡罪皆死刑,至于边际惩罚的问题,死刑也可以分级啊,留全尸的绞刑,然后是枪毙,斩首,凌迟,铁处女什么的都可以。对于犯罪者,留在监狱服刑,意味着国家机器需要拿出来一部分物质资源去维持这个监狱,这只能构成更大的资源浪费,我们为什么要养活坏人?
	
	\subsection{如何评价最近媒体关于「阴柔」现象的讨论?如何看待「阴柔美」审美文化的流行?}
	
	社会达尔文主义尊重人类的本质,人类的本质意味着男性必须有阳刚之气,如果男性女性化,那就是对自然性的异化和扭曲。
	
	\subsection{华盛顿大学有没有能哭的地方?}
	
	凌晨一点,在ode三楼,quiet study外面,小声哭。
	
	\subsection{为什么很多中国人鄙视受过高等教育的西方「白左」?}
	
	看了一下前排的其中几个答案,感觉这就是中国人特有的走极端吗?要不是“圣母心泛滥”(带引号的)“白左”,要么就要当杀光MSL的右逼蝗纳?都是弱智还要争谁智商高?“白左”喜欢治标不治本然后把自己搞的一屁股灰,右逼蝗纳喜欢头痛医头然后魔怔社达,双方都对着对面贴满标签然后疯狂攻击?这是什么逻辑?“白左”的一种睿智逻辑就是“全世界不分先进落后”,这句逻辑根本不能自洽,然后说直接侵犯人权的瓦哈比也是合理的,直接和白左的“人权”相悖;右逼则是根本没有逻辑,感觉好像是为了反左而右,比如瓦哈比和恐怖分子反人类-穆斯林是人类之癌、应该奥斯维辛,比如黑墨种族现在比较穷犯罪率就高-黑墨是劣等种族,不知道这种以偏概全又是哪里来的。作为一个ultra-progressive,我认为所有人都应该一视同仁,绝对不能也没有必要为了政治正确而偏袒某一个群体,至于家庭这些封建私有制的产物,让它解体去吧,顺便说右逼的很多政治正确更让人无语,比如不能堕胎什么的。更重要的就是,授人以鱼不如授人以渔,反正我认为不如把所有家庭全部解体一遍吧;顺便说,我认为对于某些所谓施暴者,可以效仿美国某监狱,抓起来然后教写程序,反正无论如何那个监狱的二次犯罪率基本为$0$。至于真的罪大恶极的,心理变态的,这种在任何群体也是极少数,依法处置就完事了。
	
	最后再说一下我的观点,白左死妈,右逼更死妈。
	
	\subsection{你为什么会退出 B 站?}
	
	因为这是个萌蛆网站。
	
	我前几天为了方便高雅创作弹幕黑屁搞了一个砒站号,而且还勉强升了级。但是砒站疯狂推送各种各样的萌蛆弱智纸片人游戏广告,叫人看着难受;谔次元这种野蛮费拉的东西,属实令人作呕。再说审核机制,这个审核也是效率奇差无比,慢的要死。而且还有一大堆“清规戒律”,好像谔次蛆还是个宗教一样的,不允许人抛弃道德进入存在的都是屑运营。
	
	\subsection{肥宅真的认为游戏比女朋友重要嘛?}
	
	玩游戏没意思,找女朋友更没意思,他人即地狱,告辞!
	
	\subsection{为什么崩坏三的角色全是女的?}
	
	纸片人劣等游戏日常物化女性。
	
	\subsection{哔哩哔哩用户都有哪些奇特行为?}
	
	蛆站小将是人类智商分布曲线以及脑容量曲线的最左端,所以什么东西都有是很正常的。所以这些人喜欢精分碍国、精分键政指点江山,喜欢舔劣质消费主义纸片人还不以为耻反以为荣。
	
	\subsection{人类的最后一项发明可能会是什么?}
	
	AI和VR,前者为现实代替人类,后者为人类代替现实。
	
	\subsection{为什么现在有一些初高中的孩子越来越不爱惜自己的身体?}
	
	还是现在太开放了,你看看各个公众号营销号,给早恋行为洗地的有多少。早恋本质上要被禁止就是有其合理性的,我认为这种现象发展到现在,恢复之前的严格极端反早恋制度用处也不大了,必须立法处置,用瓦哈比、孔儒、中古的方法来制约一切形式的早恋行为。
	
	\subsection{2018 年日本动画你印象最深刻的一幕是什么?}
	
	哪个刚果农奴邀请的我?
	
	\subsection{有哪些毁掉一个游戏的败笔?}
	
	我感觉前段时间很火的独立游戏《中国式家长》算一个。虽然确实挺反应中国教育的,但是很明显,败笔实在太多,无限的无用的CG剧情,萌豚化的画风,以及机械的结局$\cdots$各种地方都透露出了这个游戏制作组的低智和低质。电子游戏是人类的第九艺术,既然是艺术,就要有更深刻一点的含义在里面,而不是为了迎合大众爱好或者凑合社会热点而去哗众取宠。
	
	\subsection{假如杀掉一人便可挽救全体人类的性命,该不该杀这个人?}
	
	典型弱智迫真哲学悖论,因为大前提就是不存在的,什么情况下会出现这种假如?天灾不是杀一个人能解决的,人祸谁干的找谁。
	
	\subsection{把孩子送去让杨永信电击「治疗」的家长有后悔过吗?}
	
	楼上各位还在考虑人性,但是事实上,被送进网戒中心的人已经被迫害到失去人性了,他们出来之后只能做自己这种畜生不如的极权父母的人格依附品了,就是1984里面的温斯顿,最后爱上老大哥。
	
	\subsection{你为什么反感勃学?}
	
	新文明勃蜜,先占坑,学习勃人药,,,\footnote{原文如此。}
	
	\subsection{如果从一出生就学习 C 语言,并通过阅读代码对话,会把 C 语言当成母语吗?}
	
	死循环的时候会体现人类的本质
	
	\subsection{家长要在孩子面前表露出「赚钱辛苦、我们家不富裕」这样的态度吗?}
	
	感觉有几个社会定律:
	
	1. 每个人都会更希望向上看,这山望着那山高,这种叫做所谓上进;在资本主义社会,这个向上主要是财产意味上。
	
	2. 每个人都会对比自己强的感到不平衡,从而产生各种痛苦的情感;
	
	3. 孩子的处境起源于家长的处境。综上所述,丁克不婚是资本主义社会的最优解。
	
	\subsection{深夜给了我浅色的床单,这是什么意思?}
	
	深夜给了我浅色的床单,我却在上面拍摄创人药,,,\footnote{原文如此。}
	
	\subsection{如何评价如今OI/ACM圈存在物化女性倾向?}
	
	不钓鱼,认真回答一波。
	
	我个人认为,倒是也不算物化女性,只不过确实不算男女平等。究其原因,有下面几点:
	
	1.传统反动封建的男女有别观念让学计算机的女性缺少,而女性的缺少导致男性群体对其中少数女性的异化:
	
	我不是很相信ACM/OI这种圈子里面有类似于戒色吧吧主刘欣佛祖一样的仇女病人,也没有某些农村常见的那种活在2018年而思想还在1018年的那种老顽固,甚至连崇尚武力的民族主义菟等派系观点都不多;事实上,高技术工作者的天然意识形态是偏向于温和社会主义的进步主义。所以,ACM/OI圈的异化女性并不是ACM/OI选手本身的“顽固保守封建”这种原因,接下来继续分析。
	
	人是社会构造的产物,是社会动物,也就是说社会的规则每个人都要去遵守,无论这种规则是否合理;任何时候打破规则,虽然并不是法律规则那种绝对不可违背,但是一样会被孤立、被议论,没有人喜欢这样的情况。所谓他人即地狱,每个人都要在他人面前表现出来自己的正道直行。所以,这种传统建构的社会规则会对人的行为产生极为深远而又无法避免的影响,对于ACM/OI圈子来说,体现就是男多女少。
	
	每个人的思想都是从小形成,而社会环境给每个人思想里面留下的就是男性要如何如何,要感兴趣于什么,女性要如何如何,要感兴趣于什么,这就导致了一种隐性的规则,如果打破则意味着谴责和孤立。所以,计算机这种方向的刻板印象让传统社会有了“计算机是给男生学的”这种不合理规则,而这种规则导致了进入这个圈子的女生被视为异类,尤其是传统主义孔儒思想盛行的东亚地区,一个人进入异性居多的圈子里面,就会被潜意识的默认为“渣男”“渣女”,于是自然成为被消费的对象。
	
	至于男性ACM/OI选手本身,也是这样规则的受害者,大家都正是年轻气盛,希望找女朋友,希望谈一场刻骨铭心的恋爱,而又希望互相熟知;这种情况下,想象一下你混的圈子,男女比例9:1,出现一个女生,自然会导致某种性幻想或者关系妄想,会兴奋,会希望自己就是她男朋友,想象一下周围人那种羡慕的目光$\cdots$于是异化就出现了。
	
	2.ACM/OI选手和“二次元”等现代反进步低俗文化圈子重合度较高首先我评论区里面可能会有一大群二痴子来喷。请喷。随意喷。事实不会由于你们的拒绝承认而改变,不要试图像鸵鸟一样钻在沙子里精神胜利。“二次元”的绝大多数作品,(事实上个人认为叫做“产品”更好,用“作品”一词是对该词的侮辱)都体现了一种物化女性的价值观。(当然可能对于这个“二次元”的理解所有人之间都有歧义,我现在取其中比较狭义的意味,就是指类似于现在的日本流行ACG的各种产品,典型代表:lovelive,VOCALOID,fgo等)
	
	有些人会问:“里面角色都基本全是女的,为什么是物化女性的价值观?”很简单。这些产品都利用的是普遍男性的欲望:性欲、征服欲、保护欲等等等。这种欲望体现在现实,就是物化女性。被偶像化(广义的,指经过资本主义社会的一系列商业炒作而成名)的女性(真实或虚拟)在被炒作成名成为公众人物的同时,也相当于被作为待价而沽的货物,放在那里然后让男性(也包括少数女同性恋或者具有类似于斯德哥尔摩综合征心理的女性)去消费。当你氪金抽卡来给你纸片人老婆穿上花嫁的时候,你在考虑物化女性吗?你没有!你自己并不认为这有任何不妥,有任何反进步。
	
	“二次元”文化产生于现代的日本。现代的日本社会,正是传统孔儒文化和现代商业消费主义的结合,而这种结合的后果就是全体人类的物化,而女性的物化更为明显。在日本,我在上面说过的这种人际关系的道德绑架更加严重;可以参考一下参演了cookie企划的姐贵们,她们最后的下场不是被迫退网回归虚无就是破罐破摔去inm营业,以至于像YMN都半只脚踏进AV界了。所以,在日本产生的二次元也处处体现这样的人际关系;一个人一旦被孤立,其痛苦要远大于各种其他的痛苦,以至于死亡,这也是为什么日本的自杀率居高不下。
	
	言归正传,在这种条件下产生的这种消费主义以盈利为目的的文化,自然不会进步到哪去,总体基调是反动的,就算是可以去描写所谓的“进步”,最后的结果也是反动的。而ACM/OI选手由于忙累与圈子通常的狭窄,通常来说只能借用这样的庸俗低级反动的文化垃圾来消遣时间、发泄某些方面的欲望,由此在他们心里的底层种下了异化女性的种子。
	
	3.网络前沿普遍的异化女性思潮,以玩笑话体现出来,接受度很高;所谓的社会精英,即使是刻意“批判”,也最终还是口嫌体正直地成为了帮凶。
	
	没错,我也是帮凶之一,我承认;我没有管好我自己,这一点我是有罪的。但是有罪的不止我一个,当有罪的是大多数的时候,有罪也就是无罪了。整个网络,在ACM/OI众常混的圈子之中,普遍都是男性远多于女性的;这也就意味着,所有的ACM/OI众混的网络圈子都有第一种现象,虽然大多数情况下体现出来的是玩笑话,典型的这种玩笑话包括:
	
	“三年起步终身阴影”系列;
	
	“调戏”“RBQ”“援交”系列等;
	
	等等等等。这种词汇语句,作为玩笑话屡见不鲜,而且事实上也没有人把这种话当回事;但是对潜意识的影响是潜移默化的,在网上看到的CS相关妹子一般都比较“萌”(事实上“萌”也是非常反动的,这个之后我再开些文章写再讲),所以这些人经常去用上面这些语句来评论,反正“没有损失”。但是潜意识里面,在现实中看到的时候,还是会条件反射式的兴奋,而这种兴奋则是一般我们看到的所谓物化/异化/不平等化女性的体现。
	
	但是另一方面,你说我去拒绝“二次元”,可以做到,我就保持了下来;拒绝网络前沿的低俗信息,我感觉要做到太难了。人生在世,无聊、苦难、失败占据了绝大多数,如果仅有的休闲时间非要去故作清高地修心养性、参禅悟道,未免太痛苦了。所以这种情况还是不容易改变的,唯一能做的就是强自抑制,强行压抑自己,但是在深夜的浅色床单上,自己还是忍不住对那些平时嗤之以鼻的“自我物化的”caoser等发情。同样的,希望更多人去彻底拆穿这个社会,破旧立新,只能是这样他力本愿的了。毕竟,我自己是走不出我自己的道德圈的,否则我一定受不住各方面的所谓道德谴责;更何况,CS方向也确实是我自己的爱好,这当然意味着我几乎无法出淤泥而不染。嗟乎!思想愈发混乱,就此停笔。
	
	利益相关:生理异性恋、社会构建无性恋;很久以前摸过一点点OI的边缘但是学不会,现在上大一,是个考虑摸ACM的边缘,但是大概也学不会的彻头彻尾的失败者。我到哪里,哪里就是带专。
	
	\subsection{如何看待女权称“男性不应该和女性平起平坐”,并且被一大堆微博女权点赞?}
	
	这些“女权”大概都忘记女权最早是干什么的了。女权最早要求的就是平等,要在社会的一切领域给两种性别创造同样的机会和权利,要消灭两种性别在社会各个方面上单纯因为性别导致的差别,现在倒好,把二百年前的保守派男权那一套反加给男性身上,要么是因为受伤害或者某种神经生理的问题而出现的仇男症状,要么是别有用心试图通过激进言论吸引粉丝盈利甚至刻意污名化女权,总之非蠢即坏。
	
	\subsection{你怎么看「男子 20 年后拦路连扇老师耳光:还记不记得我?」?}
	
	仇恨,冤冤相报,自然是不行的。复仇,自然是痛快,但留下的副作用也不能被抹除。只能说,该发声的时候,必须要发声;这样的老师,被打只能说活该,但是打人的学生这种方式也着实并不很合理。我的建议是这个老师就算翻旧账也要从重处罚,这个打人的学生从轻处罚。
	
	\subsection{知乎数学大V dhchen的水平和知名数学民科Strongart的数学水平孰强孰弱?}
	
	陈德汗即使变性也不配给于志成当女仆,不多谈!
	
	\subsection{假如把一个人粉碎成原子再组合,这个人还是原来的人吗?}
	
	看了半天,只能说逼乎人均民科+网哲。一个比一个睿智。粉碎成原子然后再按照原来方式排列,根据海森堡测不准原理,这就已经不可能了,因为你要储存的信息包括了每个量子内部的所有信息,否则意识就不连续了。
	
	\subsection{为什么中国女权主义者不去获取中立人群的支持而是一棍子打死?}
	
	挂着女权名号引战钓鱼或者专门招黑的弱智“费米纳粹”才不会争取任何人,这些人也就能嘴炮两句,越多人骂他们他们越高兴;推行口红色号论之类的那帮事实上维护父权制的伪女权也不会争取任何人,因为他们自己就是一群被资本主义异化了的人;真正的女权政治坐标至少是社会民主主义以左,他们也会自主思考,这种人其实不少,只不过通常很少上网发言,上网发言也没人看,因为消费主义当道碎片文化盛行的时代只有那种弱智一样的痛快话才能博取关注被网民看到。到最后,你看到的就只剩下引战弱智和反向卫道士了
	
	\subsection{如何看待中国2018年性别差距排名从100位下滑至103位?}
	
	田园女犬功不可没,毕竟是他们让女性停止担当社会工作、无需在教育和思想深度上自我提升,而蹲在家里当大爷,从而维护男性在社会工作以及政治科技等方向的绝对统治权的
	
	\subsection{对B站乃至日本ACG抱有极大敌意并视其为眼中钉肉中刺的爱国系贴吧和电竞系贴吧有对B站造成过实际破坏吗?}
	
	一种是碍国人士,还有一种是我这样的精神洁癖。
	
	\subsection{被老板剥削,如何对他实行无产阶级专政?}
	
	不如zs,反正被剥削是你的命,谁叫你生在这种吃人社会,或者你和于志成一样自我炒作加啃老消极反抗也行,就带国这民智水平,要革命还是等个一二百年让各路思想先由各类民科网哲键政初生演替一遍再说吧
	
	\subsection{人有可能完全摆脱性别概念,以中性或者无性别的身份活下去吗?}
	
	谢邀。
	
	以性别为基础的stereotype在全社会非常普遍,这不是一天两天能够改变的,取消性别概念对于个人发展和个人自由的影响需要几代人的努力。毕竟现在而言,保守主义仍然市场广泛,男女有别仍然是属于公理常识。除非有一天人类可以彻底解体家庭概念和性别概念,并取消依赖于体能的体力劳动,否则人类不可能做到彻底的消灭性别概念。这需要无神论、生物学、社会学、计算机科学的统合推进,而以现在人类的思想水平和科技水平,还远远达不到。毕竟,精神中古人还多的是,很多人活在现代,但是却是中古的虔信之众,用瓦哈比、天主、印度诸神或者孔儒的名号,亦或是所谓的“祖宗之法不可变”,来障自己之眼,以求不适宜的规则继续存在。
	
	\subsection{在中国,到底什么算极右?}
	
	1. 菟、岁静婊、粪青
	
	菟和岁静婊维护的是一个极权主义和民族主义的政权,维护的是类似于弗朗哥和皮诺切特的统治,拥护的是一个社会达尔文主义的社会,所以绝对属于极右,但是应该算相对温和的极右;粪青更过分一点,宣扬的种族主义更加充满暴力的法西斯气息,充满了仇恨,属于激进的极右
	
	2. 蝗汉
	
	蝗汉要求国家成为汉族的种族单一的国家,这已经符合极右翼民族主义的特性了,再加上其要求恢复本质上属于封建反动文化的各种古汉文化(汉服等),当然属于极端退步的极右
	
	3.德棍黄纳、昭和精日
	
	这个不用我说了,能舔德三小胡子或者带政逸赞会的,不是极右也没有了
	
	4.远邪姨学小鬼
	
	姨小将搞民族发明学,然后就想裂土分国、联省自治,虽然你姨实质上还是拾斯宾格勒之牙慧,但是你姨小鬼已经开始发明一套奇妙深刻的迫真种族主义了,凭这种奇妙深刻理论,当然是极右5.成功学家成功学家发明的如何成功理论无非就是宣扬社达,不成功的都是社达竞争失败,你要如何在社达竞争中存活以成功,鼓吹无条件的社达,当然是极右
	
	\subsection{你现在是否处于「自杀式单身」?}
	
	自杀是唯一严肃的哲学命题--加缪
	
	每个人每一天都在慢性自杀,所以单身必然是自杀式单身,恋爱也是自杀式恋爱。
	
	独身主义是一个人保持理性的必要条件,在不单身的情况下保持绝对的理性和思想自由以及对外交往当中的博爱是不可能的。
	
	对于一个精神高贵的人来说,没有任何理由要去追寻爱情。
	
	社会构建之中,大多数人其实都是或者应该成为无性恋,因为我们是失败的,失败人自然没有资格和权利留下后代,而如果强制假装自己是异性恋而去恋爱,去留下后代的话,无非就是让下一代继续承受痛苦。毕竟失败是必然痛苦的,而失败是无法避免的必然。
	
	一个一百本一万本的失败人士,自然要求最优解,那就是苟活,向死而生,既然如此,还有所谓之自杀式单身或者自杀式恋爱吗?
	
	May we live long and die out.
	
	\subsection{性别之间是否应当有界限?}
	
	泻药
	
	取消性别概念刻不容缓,这也就不是所谓保持距离与否,而是彻底对性别概念进行取消,取消每个人的性别属性,这需要机械飞升和生物飞升大道的实现。
	
	低劣的文明之上的家庭结构严重制约了人类总体生产力的进步与发展,所以应该彻底解体家庭、砸烂伦理学,进行一次彻底的社会革命。
	
	这需要的基础是体外直接克隆的生物学或者是意识植入机械的计算机科学。自然科学的基础一旦达到,就是性别概念彻底毁灭之日。
	
	以上。
	
	\subsection{为什么专科不受人待见,专科生真的没有出路吗?}
	
	专科还是zs吧,不用说专科,浙大水平的三本都没出路只能zs,我现在在一个连浙大都不如的垃圾百本,我只能怪我不好好努力,在这样的百本里面也当不了最强的,我是傻逼。
	
	\subsection{复读 8 年考上清华大学到底值不值得?}
	
	“对于那些进入竺可桢学院就觉得自己其实没有亏的人,我只能这么说一句:眼光还是太短浅了。超级大牛不算。但我们都是平凡人。我个人认为,浙大拿专业第一的实际市场价值,不如清华大学的专业前15-10。我给你们的建议是:如果你是考败来浙,建议复读。只需要1年,可以弥补你日后3年,5年都无法弥补的遗憾。我没有复读,我没有看到更广阔的天空,我没有更多的人脉,我不能成为高晓松的校友,我不能和街上的大神们有说有笑,我是傻逼。”——勃勃
	
	我现在在国外百本垃圾大学当韭菜。我感觉,这样着实值得。
	
	我自己复读了一年,至少没有至于去万本,而来到了这个百本。但是,我在这里的学习机会仍然有限。如果我再复读些年,进入清北或者藤校,我不会成为这样子的韭菜一个。我是失败人,因为我没有复读。
	
	\subsection{如何看待央视点名批评哔哩哔哩?}
	
	蛆站该杀,本质上蛆站是一个基于恋童癖和物化女性实质的反动“二次元”ACG的垃圾网站,我使用之的唯一理由是缺乏替代品,因为其他国内中文网站的广告过多或者是投稿门槛过高严重影响正常视频的观看创作的体验。只要有一个新的网站,拒绝萌豚二次元,拒绝恋童擦边球,我会毫不犹豫地带头抵制蛆站。
	
	\subsection{如何看待Vtuber风潮?}
	
	二次元没有一个是无辜的,消费主义继续打擦边球出售物化人性,还有,舔vtuber的二次蛆也都是够能双标的,三次元女主播就恶俗,换成纸片人就高雅了,我呕呕
	
	\subsection{高校大门是否应该敞开?}
	
	原则上,应该开放。因为毕竟国内的对全社会开放的学习场所真的太少了。
	
	但实际执行过程中,必须考虑到具体情况。
	
	首先必须有严格成文的管理规定,能在什么时间做什么都规定清楚,对于那些想要搞事的扰乱秩序的打搅学习的,应该得有人去管,比如那些广场舞大妈什么的绝对不能允许在操场上撒野。校警要管事。
	
	第二是学生食堂等可以考虑用双轨价格,非本校师生一律按照相对高一点的价格,避免大群大爷大妈哄抢。
	
	第三是确定一套可行的面向社会选课系统,来“蹭课”的人也得交费,也要记录出勤、交作业、考试,课过了拿盖章证书。那些个来慕名蹭课甚至是去拦住教授要签名或者干什么的那种人趁早远点去。
	
	\subsection{顶尖的大学应不应该招收这样的学生?}
	
	我个人认为,这种学生是不应该被招的。因为trivial solver确实是缺乏创造力和实践能力的。
	
	一定的成绩对于在大学的学术成功是必要的,所以要设定一个相对灵活的最低分数线。然后对于线上的学生,应该用各种指标来衡量他们的能力,比如他们在一些学科的获奖或者一些可以证明的自主学习自主创造,然后选择性招生,这样都是可以的。
	
	现在的内卷化之体现就是一大群人为了挤一个单一制的高考独木桥,在无数极端trivial无用的各种题各种卷子当中消耗了自己几乎全部的精力,但是其中所涵盖的知识点却只有那么多。这种单一的高考的制度下,没有学生会被鼓励学习更高一层次的内容,虽然相比于在极端复杂而又没有实际知识技能意义的题上面浪费时间精力来说去学习更高层次内容是更有益的。
	
	\subsection{性别的差异在于生理基础即身体结构,还是在于思想?}
	
	泻药。
	
	这种思想我们无法直接判定正确或者错误,但是可以判定认同或者不认同。我的观点是,部分认同。
	
	性别概念目前来说仍然是基于生理结构的,但是也具有社会构建,并且主要的思想层面的性别概念来源于社会的构建。在目前的社会构建之下,思想是会随着性别不同而不同的。性别概念需要被取消,虽然暂时我们无法彻底消灭生理性别和有性生殖,但是我们可以在意识形态上改进,消灭实际社会生活中的性别差距,从而淡化性别概念对个人的影响。
	
	以上。
	
	
	\subsection{你觉得,你的单身有那些优势?}
	
	1.自由,在于思想层面的自由。在不单身的情况下,难以维持真正的理性,去思考和面对,去出世以求窥得世界的面貌,然后去思考世界之意义与发展的方向。
	
	2.决策。在独身的情况下,所有规划由自己决定完成。在两个人的世界里,55扯皮的情况特别多,民主做不到,君主就意味着不平等。
	
	3.责任。独身情况下,不具有婚恋之道德责任,需要面对的社会道德绑架和各种道德危机也少得多。
	
	4.博爱。在独身情况下,一个人可以不分性别、保持距离地与任何人正常相处,去交流、争论、产生思想共鸣和双赢博弈。而如果不独身,那么在与人相处当中就会非常受制,必须停止和异性相处。
	
	5.经济。所谓婚恋是合法的长期卖淫,一旦脱离独身就意味着具有很严重的经济上的相互关系,这种经济上的相互关系对于我们绝大多数绝对失败的人来说就是纯粹的双输博弈。
	
	\section{强权压制下的武学精华——退学人13}
		\subsection{那些没听长辈劝告,与穷男朋友结婚的人现在都过得咋样了?}
			不用听信楼上的,婚姻是合法的长期卖淫,社会达尔文主义的准则其实是全社会想要回避也回避不了的。浪漫主义的结果只有悲剧。现在,出身已经决定了一个人的60\%

		\subsection{不看抖音、快手,不玩王者、吃鸡的是什么样的男人?}
			抖阴快手本质弱智农民,亡者农药也是,至于吃鸡的话不尽然,不过我个人也是3/4的这种人,我个人认为人类已经没救了,嗯就是这样

		\subsection{一个真正的无神论者是什么体验?}
			“我们将下地狱,解放地狱,而后解放天堂。”——克洛索斯利亚大统领 提琳戈龙(2089-2119)

		\subsection{女生到底有多在乎男生的身高?}
			本人男,净身高大概174,注孤生就完事了

		\subsection{如何评价北京卫视哈佛毕业女学生的演讲?}
			我们的国家是全世界最伟大的国家,中华民族是全世界最伟大的民族,西方世界都是地狱,等待我们伟大祖国前去解放

		\subsection{为什么感觉现在的大学男生都不会追女孩子了?}
			都8102年了还有睿智传统主义者和乌托邦弱智反对“婚恋就是合法的长期卖淫”这个命题?xswl,作为一个自由意志人,独身主义万岁

		\subsection{如果你的孩子是同性恋,你会持怎样的态度?}
			孩子还小,打死就好了

		\subsection{有哪些让你觉得恶心的恶俗歌词?}
			明星没有一个是无辜的

		\subsection{为什么这几年男生都不愿意去追女生了?}
			女犬婊把自己人格作为附庸而财产作为主导,没有男人会想养一个没有思想只会花钱的吉祥物,可惜现在这种吉祥物化,或者说女性自我物化思想成了主流

		\subsection{为什么国内网络称联合国常任理事国为“五大流氓”?}
			小粉红国流放,殊不知大国其实已经走到了帝国主义时代,已经走到了剥削原料产地倾销商品的普鲁士资本主义时代

		\subsection{吃鸡游戏故意杀队友的人是什么心理?}
			我遇到过,满嘴四川腔脏话的睿智,一口一个sb,没办法,这种人估计就是小学初中的一些睿智,被快手之类的影响了,然后自以为自己是什么为所欲为的社会暴徒,其实还不是一群hape

		\subsection{那些说别人是easy girl的人是什么心态?}
			\textbf{原文.}
			看了楼上几个人回答,建议出台《种族纯粹法》,禁止一切外国人进入大国

			\noindent ——分割线——

			\noindent 亲外和排外就是两种不同的弱智,前者给了后者借口,而后者则可以激发前者扣帽子,于是这世界上有了极左和极右,esg和蝗汉,慕洋犬和碍国贼。\\

			\textbf{更新.}
			请所有新纳粹分子停止在我的评论区叽叽歪歪,亲外弱智死妈在于亲外弱智不分黑白亲外,同样排外弱智死妈在于不分黑白排外,可惜碍国人士的政治正确可只有排外,要疯狂扣帽子,要坚信反对东京大屠杀都是精日,否则就是圣母婊,呵呵。敢问评论区里面的爱国者敢不敢当个布雷维克?为最伟大光荣的民族正名?去杀洋鬼子,去杀es婊,去杀亲外圣母豚,你们不敢!顺便说挪威圣母婊给布雷维克好吃好喝养着还只有21年的刑期,可惜大国可没有这种圣母婊罩着你们。本质上无论是洋垃圾还是土垃圾都是垃圾,无所谓高下,只可惜亲外和排外废物都喜欢选择性眼瞎,没办法,叫不醒装睡的人。甲乙两人在路上看到两坨散发臭味的屎,甲说:“左边那坨屎是张村的人拉的,我们王村的人屎才不臭!张村没有一个是好东西!”乙说:“右边那坨屎是我们王村的人拉的,这臭味还不都是我们王村的劣根性!”

		\subsection{为什么男生坐着的时候喜欢叉开腿?}
			\noindent 不请自来 \\

			\noindent 胖是原罪。长得胖,腿非常粗,不分开,也像初中政治书里面一样并拢,我撑不了3分钟。\\

			\noindent 自然分开三四十度,坐着自然舒适稳定。

		\subsection{作为男生,你的择偶标准是什么?}
			只有一个标准,那就是不存在。

			\indent 不允许我保持绝对理性和真正自由的,换言之,任何对我进行道德绑架和强加责任的事务,都通通远离我的视线。

		\subsection{求六小龄童纽约春晚网址?不知道有没有网络直播,中国春晚没猴哥,反倒纽约春晚有猴哥,哎~今年春晚看纽约?}
			你这样子是要向全世界人民谢罪的,改编不是乱编,戏说不是胡说

		\subsection{如何看待六小龄童对名著进行删减,以及青少年版名著对原作品进行删减又是否违法?}
			\noindent 戏说不是胡说,但六老师可以胡说;\\

			\noindent 改编不是乱编,但六老师有权乱编。\\

			\noindent 一千个人心目中可以有一千个哈姆雷特,但一千个人心目中只能有六老师一个美猴王,一个吴承恩。

		\subsection{bilibili(哔哩哔哩)在纳斯达克上市意味着什么?}
			意味着除了炼铜、卖银、氪金以外什么都不会的大型垃圾公司开始圈钱了,而且还有一群一群的二次蛆给这种毫无实业基础对于人类有百害无一利的垃圾公司投钱当韭菜

		\subsection{为什么大V刘宇航想要退出数学,而民工dhchen还洋洋得意地在野鸡大学上蹿下跳?}
			因为陈德汗只是一个噶韭菜的数分复读机而已,根本不是数学人,刘宇航是正儿八经的数学学者,要退出当然可以退出了

\end{document}